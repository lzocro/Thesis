%
%
%
%===============================================================
%Packages 
%===============================================================


%Fonts
\usepackage{algorithm2e,listings,}

%Palatino
\usepackage{mathpazo}
\usepackage{type1cm}

%Maths
\usepackage{amsmath,amsfonts,amsthm,amsopn,amssymb,mathtools,bigints,stmaryrd,mathrsfs}
\allowdisplaybreaks

%Presentation
\usepackage{graphicx, tikz, caption,lettrine}
\usepackage{xcolor}
\definecolor{dropcap}{HTML}{b10000}

%bib
\usepackage[square,numbers,comma,super,sort&compress]{natbib}
\bibliographystyle{humannat}

\makeatletter
\newcommand{\citethm}[2][]{%
\begingroup
  \let\NAT@mbox=\mbox
  \let\@cite\NAT@citenum
  \let\NAT@space\NAT@spacechar
  \let\NAT@super@kern\relax
  \renewcommand\NAT@open{[}%
  \renewcommand\NAT@close{]}%
  \cite[#1]{#2}%
  \endgroup
}
\makeatother


%notation index
\usepackage{nomencl}
\makenomenclature
\usepackage{multicol}
\setlength{\columnsep}{1cm}
	%preamble
	\renewcommand{\nomname}{Index of Notation}
 
\renewcommand{\nompreamble}{This index orders by domain the symbols used in this thesis which have an intrinsic meaning. Unless listed as different here, notation carries over from one chapter to the next.}
	%creates groups
\usepackage{etoolbox}
\renewcommand\nomgroup[1]{%
 	\item[\bfseries
 	\ifstrequal{#1}{M}{Measure Theory}{%
 	\ifstrequal{#1}{I}{Information Theory}{%
	\ifstrequal{#1}{B}{Bandit Theory}{%
	\ifstrequal{#1}{Z}{Miscelaneous}{}}}}%
]}

%guarantees page n� are included in the citation super
\makeatletter
\renewcommand\NAT@citesuper[3]{\ifNAT@swa
\if*#2*\else#2\NAT@spacechar\fi
\unskip\kern\p@\textsuperscript{\NAT@@open#1\if*#3*\else,\NAT@spacechar#3\fi\NAT@@close}%
   \else #1\fi\endgroup}
\makeatother


%===============================================================
%Commands
%===============================================================


%paragrphs
\setlength{\parskip}{0.15in}
\setlength{\parindent}{0.3in}


%BB
\newcommand{\NN}{\mathbb{N}}
\newcommand{\ZZ}{\mathbb{Z}}
\newcommand{\RR}{\mathbb{R}}
\newcommand{\PP}{\mathbb{P}}
\newcommand{\EE}{\mathbb{E}}
\newcommand{\II}{\mathbb{I}}

%operators
\newcommand{\de}{\mathrm{d}} %integration d
\newcommand{\cov}[1]{\, {\rm cov}\left( #1 \right) } %covar
\newcommand{\var}[1]{\, {\rm var}\left( #1 \right) } %var
\newcommand{\argmax}{\text{argmax}} %argmax
\newcommand{\argmin}{\text{argmin}} %argmin

\DeclarePairedDelimiter{\ceil}{\lceil}{\rceil} % ceiling

% Theorem definitions %%% Review numbering system

\newtheorem{lemma}{Lemma}[section]
\newtheorem{theorem}[lemma]{Theorem}
\newtheorem{proposition}[lemma]{Proposition}
\newtheorem{corollary}[lemma]{Corollary}

\theoremstyle{definition}
\newtheorem{definition}{Definition}[section]

\theoremstyle{remark}
\newtheorem{remark}{Remark}[section]
\newtheorem{example}[remark]{Example}

%listing code set up

\lstset{ %
basicstyle=\footnotesize,       % the size of the fonts that are used for the code
backgroundcolor=\color{white},  % choose the background color. You must add \usepackage{color}
showspaces=false,               % show spaces adding particular underscores
showstringspaces=false,         % underline spaces within strings
showtabs=false,                 % show tabs within strings adding particular underscores
frame=single,           % adds a frame around the code
tabsize=2,          % sets default tabsize to 2 spaces
captionpos=b,           % sets the caption-position to bottom
breaklines=true,        % sets automatic line breaking
breakatwhitespace=false,    % sets if automatic breaks should only happen at whitespace
escapeinside={\%*}{*)}          % if you want to add a comment within your code
}