%\lettrine[lines=4,slope=1pt]{A}{ll} 
 %\lettrine[lines=3,slope=1pt]{\textcolor{red}{A}}{n} 

\chapter{Stochastic Bandits}\label{chap:2}

\lettrine[lines=4]{\textcolor{dropcap}{M}}{odelling} complex, partially unknown real world systems is typically done by replacing unknown mechanics with random approximations. This is the case of the one-armed bandit for instance, one gambles against a seemingly random slot machine. The slot machine however, is wholly deterministic. It is therefore natural to study first the problem of sequential action allocation in a stochastic environment. In this chapter, we will establish some mathematical foundations which we will use to evaluate the performance of algorithms in chapter \ref{chap:algs}. We will begin by outlining mathematical concepts which will be needed to study the performance of bandit algorithms, and then by characterising the general properties of the stochastic bandit problem.

%
%
\section{Mathematical notes}\label{sec:Maths}
%
%

\lettrine[lines=3]{T}{he} main quantity of mathematical interest in this thesis is the concept of regret. Like in all other machine learning problems, bandits are motivated by the optimisation of a loss function, equivalent to the maximisation of obtained rewards. As the rewards are random variables, it will be natural to consider the expectation of accumulated rewards as a function of time. The question is what to compare rewards to, in order to obtain a meaningful loss which the algorithm can learn from. This will motivate our definitions of regret, after which we will show a quintessential theorem called the regret decomposition identity. After this, we will turn our attention to the problem of concentration inequalities on the mean of certain random variables. We will define sub-gaussianity, then show two important concentration inequalities.  

%
\subsection{Regret Properties}\label{subsec:regret-prop}
%

\par What is referred to as the \textit{regret}�can be confusing, as such we will begin by clarifying its exact definitions used in this thesis. Afterwards we will formulate the regret decomposition identity, which will prove to be a fundamental result. 
The regret is a random variable, $R_n$, defined as the difference between repeating the action of highest summed rewarded and the received sum of rewards\cite{bubeck:2012}. In a way, the regret represents the loss in the bandit setting relative to a perfect information setting. Mathematically:
\[ R_n = \max_{i\in\mathcal{K}}\sum_{t=1}^n X_{i,t}-\sum_{t=1}^nX_{I_t,t}\,. \] 
It would be natural to examine the \textit{expected regret} $\mathbb{E}[R_n]$, where the agent competes against the expected value of the maximal reward obtainable by playing the same arm repeatedly. In practice however, we will tend to examine the weaker notion of {\em pseudo-regret}, as defined in definition \ref{def:PRSB-1}, where one competes only against the sequence which is optimal in expectation. The pseudo-regret is colloquially referred to as the regret in literature, for simplicity, and will sometimes henceforth be referred by this simplification. The reader should, however, bear in mind this terminological sloth. 

\begin{definition}[Pseudo-Regret of a stochastic bandit]\label{def:PRSB-1}
In a stochastic bandit with $K$ arms, for round $0<n\leq N$, we define the pseudo-regret as:
\[\bar{R}_n=\max_{i\in\mathcal{K}}\left\{\mathbb{E}\left[\sum_{t=1}^n X_{i,t}-X_{I_t,t}\right]\right\}\, . \]
\end{definition}

\par It is straightforward to reorder this definition into a more tractable form by defining $\mu^*=\max_{i\in\mathcal{K}}\{\mu_i\}$ to be the expected payoff of the optimal arm.

\begin{definition}[Tractable pseudo-regret of a stochastic bandit]\label{def:PRSB-2}
\[\bar{R}_n= n\mu^* - \mathbb{E}\left[ \sum_{i=1}^n X_{I_t,t}\right]\, . \]
\end{definition}
\begin{proof}
Starting with definition \ref{def:PRSB-1} we rewrite the formula for $R_n$: 
\begin{align*}
\bar{R}_n&= \max_{i\in\mathcal{K}}\left\{\mathbb{E}\left[\sum_{t=1}^n X_{i,t}-X_{I_t,t}\right]\right\} \\
&= \max_{i\in\mathcal{K}}\left\{\mathbb{E}\left[\sum_{t=1}^n X_{i,t}\right] -\mathbb{E}\left[\sum_{t=1}^nX_{I_t,t}\right]\right\}\\
&= \max_{i\in\mathcal{K}}\left\{n\mu_i\right\} - \sum_{t=1}^n\mu_{I_t}\\
&= n\mu^*-\sum_{t=1}^n\mu_{I_t}\,. 
\end{align*}
\end{proof}

\par These two definitions are all we need to introduce the main result of this section, whose proof will be immediately given thereafter. 

\begin{theorem}[Regret Decomposition Identity]\label{thm:RDI}
In a stochastic bandit with $K$ arms, for $0<n\leq N$, let $\Delta_k=\mu^*-\mu_k$ be the sub-optimality gap for arm $k$, and let $T_k(n):=\sum_{t=1}^n \mathbb{I}\{I_t=k \}$ be the random variable counting the number of times arm $k$ is chosen in $n$ rounds. We can now decompose the pseudo-regret in terms of $\Delta_k$ and $\mathbb{E}[T_k(n)]$:
\[\bar{R}_n=\sum_{k\in\mathcal{K}}\Delta_k\mathbb{E}[T_k(n)]\, .\]
\end{theorem}

\begin{proof}
Recalling from the tractable pseudo-regret, we will re-index the sums from the number of rounds to the number of arms:
\begin{align*}
\bar{R}_n &= \sum_{t=1}^{n}\mu^* - \sum_{t=1}^{n}\mu_i \\
&=\sum_{t=1}^{n}\left(\mu^*-\mathbb{E}[X_{I_t,t}]\right) \\
&=\sum_{t=1}^{n}\mathbb{E}[ \Delta_{I_t} ]\\
&= \sum_{t=1}^n \sum_{k\in\mathcal{K}} \mathbb{E}[\Delta_k\mathbb{I}\{I_t=k \}]\\
&=\sum_{k\in\mathcal{K}}\Delta_k \mathbb{E}\left[\sum_{t=1}^n \mathbb{I}\{I_t=k \}\right]\\
&=\sum_{k\in\mathcal{K}}\Delta_k \mathbb{E}[T_k(n)]\, .
\end{align*}
\end{proof}

\par Theorem \ref{thm:RDI} concludes the list of regret properties we need to review, and we will now move on to a discussion of the estimation of a mean, which will allow us to derive tail probability bounds we will need.

%
\subsection{Estimation of a Mean}\label{subsec:mean}
%

\par In order to set-up algorithms for stochastic bandits it is necessary to effectively estimate the mean pay-off of each arm. Indeed, each time we pull a sub-optimal arm to refine its mean, we incur a penalty with expectation $\Delta_k$. This seems like a straightforward task, after-all there is an unbiased estimator for the mean, the sample mean, which we can adapt: $\hat{\mu}_k:= \frac{1}{n_k}\sum_{t=1}^{n} X_{I_t,t}\mathbb{I}\{I_t=k\}$. However, being unbiased is not the paramount property in the case of a bandit problem. If the estimator is unbiased but has high variance it will be difficult to determine which arm has the highest true mean. Recall that the variance (i.e. error) of the sample mean is $\frac{\sigma^2}{n}$, where $n$ is the number of samples and $\sigma^2$ is the variance of their underlying distribution. 
\par We would like to define a framework to describe the distribution of $\hat{\mu}$, which is unknown. To do so we will look at the tail probabilities $P(\hat{\mu}\geq \mu + \epsilon)$ and $P(\hat{\mu}\leq \mu - \epsilon)$ and attempt to bound them. We could use Chebyshev's inequality or the central limit theorem, but the first one is a weak bound and the second one is asymptotic which forbids its use. Instead we will define a new property of a random variable which will allow us to derive new properties about its tail probabilities. 

\begin{definition}[Sub-gaussianity]
A random variable $X$ is $\sigma^2$-sub-gaussian if it satisfies:
\[\forall \lambda\in\mathbb{R}: \, \mathbb{E}[\exp(\lambda X)] \leq \exp\left(\frac{\lambda^2\sigma^2}{2}\right)\, .\]
\end{definition}

\par Intuitively $X$ is sub-gaussian if its tails are lighter than the gaussian distribution, which is to say that it has lower tail probabilities. We follow this with some simple results about independent sub-gaussian random variables.

\begin{lemma}[Properties of sub-gaussian random variables\cite{banditalgs:3}]
Let $X$ be a sub-gaussian random variable.
\begin{list}{$\circ$}{}
\item We have $\mathbb{E}(X)=0$ and $\var{X}\leq \sigma^2$. 
\item For $c\in\mathbb{R}$, $cX$ is $c^2\sigma^2$-sub-gaussian.
\item For $X_1\bot X_2$ $\sigma_1$-sub-gaussian, and $\sigma_2$-sub-gaussian respectively, $X_1+X_2$ is $(\sigma_1^2+\sigma_2^2)$-sub-gaussian.
\end{list}
\end{lemma}

\begin{proof}\hfil
\begin{list}{$\circ$}{}
\item We consider $\lambda\neq0$, as this is a vacuous case. Note that we can expand the definition as
\begin{align*}
\mathbb{E}[\exp(\lambda X)]&\leq\exp\left( \frac{\lambda^2}{2\sigma^2}\right)\\
\mathbb{E}\left[ \sum_{n=0}^\infty \frac{(\lambda X)^n}{n!}\right]&\leq\sum_{n=0}^\infty \frac{1}{n!}\left(\frac{\lambda^2}{2\sigma^2}\right)^n\,.
\intertext{Using the second order taylor expansion, we obtain for all $\lambda\in\mathbb{R}$:}
\lambda\mathbb{E}[X]+ \lambda^2\mathbb{E}[X^2] &\leq \frac{\lambda^2}{2\sigma^2} + R_2(\lambda)\,.
\intertext{We separate cases where $\lambda>0$ and $\lambda<0$, and divide by $\lambda$. Taking the limit to $0$ gives:
\[E(X)\geq 0 \text{ if}\,\lambda<0 \text{ and}\,E(X)\leq 0 \text{ if $\lambda>0$} \Rightarrow E(X)=0\,.\]}
\intertext{For the variance we divide instead by $\frac{\lambda^2}{2}>0$, and take the limit as $\lambda\to0$:
\[\var{X}=\mathbb{E}[X^2]-\mathbb{E}[X]^2\leq \frac{1}{\sigma^2}\,.\]}
\end{align*}
\item Letting $\lambda'=\lambda c�\in \mathbb{R}$ in the definition gives the result.
\item A simple computation gives:
\begin{align*}
\mathbb{E}[\exp(\lambda(X_1+X_2))]&=\mathbb{E}[\exp(\lambda X_1)]\mathbb{E}[\exp(\lambda X_2)]\\
&\leq\exp\left(\frac{\lambda\sigma_1^2}{2}\right)\exp\left(\frac{\lambda\sigma_2^2}{2}\right) \\
&=\exp\left(\frac{\lambda(\sigma_1^2+\sigma_2^2)}{2}\right)\,.
\end{align*}
Thus $X_1+X_2$ is $(\sigma_1^2+\sigma_2^2)$-sub-gaussian.
\end{list}
\end{proof}

\par To formalise our intuition of the lightness of tails of sub-gaussian random variables we introduce the following concentration inequality, which we prove using Chernoff's method. 

\begin{theorem}[Concentration of Sub-gaussian Random Variables\cite{banditalgs:3}]\label{thm:concentrationSGRV}
If $X$ is a $\sigma^2$-sub-gaussian, then $P(X\geq \epsilon)\leq \exp\left(-\frac{\epsilon^2}{2\sigma^2}\right)$.
\end{theorem}

\begin{proof}
Let $\lambda>0$. As exponentiation conserves inequalities we have:
\begin{align*}
P(X\geq \epsilon)&= P(\exp(\lambda X) \geq \exp(\lambda \epsilon))\, .\\
\intertext{As $\lambda\epsilon>0$, we can apply Markov's inequality to $\vert X \vert$, which gives us an upper bound for the above:}
P(X\geq \epsilon)&\leq \mathbb{E}[\exp(\lambda X)]\exp(-\lambda\epsilon)\\
&\leq \exp\left(\frac{\lambda^2\sigma^2}{2}-\lambda\epsilon\right)\,.
\intertext{Now we choose $\lambda$ to minimise this bound as it holds for all $\lambda>0$. See that we take $\lambda=\frac{\epsilon}{\sigma^2}$ and thus have:}
P(X\geq \epsilon)&\leq \exp\left( -\frac{\epsilon^2}{2\sigma^2}\right)\, .
\end{align*}
\end{proof}

From theoreom \ref{thm:concentrationSGRV}, we can derive a bound for the sample mean we were interested in. 

\begin{corollary}[Hoeffding's bound]\label{cor:Hoeffding}
Let $X_i-\mu$ be independent $\sigma^2$-sub-gaussian random variables. Then $\hat{\mu}$ has tail probability bounds:
\[P(\hat{\mu}\geq \mu +\epsilon)\leq \exp\left(-\frac{n\epsilon}{2\sigma^2}\right)\text{ and } P(\hat{\mu}\leq \mu -\epsilon)\leq \exp\left(-\frac{n\epsilon}{2\sigma^2}\right)\, .\] %combine into both are bounded by 
\end{corollary}


\par This concludes our discussion of estimation of means as we have found satisfactory bounds on tail probabilities for the sample mean. We now outline the general characteristics of the stochastic bandit problem in terms of the regret properties which we will use to compare competing algorithms. 

\section{Characterising the Stochastic Bandit Problem}\label{sec:chars}

\lettrine[lines=3]{G}{eneral} results about the regret behaviour of algorithms in the stochastic bandit problem will allow us to establish benchmarks against which we can compare the effective performance of the algorithms which will be discussed in chapter \ref{chap:algs}. We will present the key behaviour of regret growth, characterising the types of policies which we will take as valid, and then proving a tight lower bound on the general-case asymptotic regret growth attainable, the seminal result in bandit theory. This first analysis will focus on what can be achieved on nice instances, the first benchmark, while the second will analyse the best-achievable worst case regret, a concept borrowed from game theory, which will be defined. This second benchmark allows us to compare performance of algorithms in the most gruelling environments, where the random distributions are virtually equivalent to playing against an adversary, bent on tricking the algorithm. 


\subsection{Optimal Asymptotic Regret Growth}

\par The key result in this section will show that any strategy which is {\em consistent} achieves at most logarithmic asymptotic regret growth in terms of the number of rounds, up to a universal constant which will make this bound tight. It is not trivial however to see why this is, nor to understand it. It is important firstly, that this behaviour is asymptotic, and must only be considered if $N$ is large in regards to the difficulty of the instance, and the number of arms. Since we must intuitively explore the whole action space to hope to find the optimal arm, algorithms often start by pulling each arm one or more times, denoted $m$. This leads to a linear regret term for the first $mK$ rounds, which does not interfere with the result. A further problem with asymptotic bounds, is that in a finite time instance, the asymptotic domain may never be reached\cite{Garivier:2016}, leading to non logarithmic growth. Notwithstanding, asymptotic guarantees are still a valuable measure for algorithms, even if they must be taken with the usual caution for asymptotic results. 

\par In deriving a general result for all acceptable policies, the next issue we are faced with is exactly how to define an acceptable policy. Of course we can eliminate trivial policies which are inapplicable in practice, such as $\pi(t)=\argmax_{k\in\mathcal{K}}\mu_k$, which requires knowledge of the best arm ahead of time, while achieving 0 regret. Thus our policy $\pi$ must not take into account the means $\mu_i$, and therefore not the sub-optimality gaps $\Delta_i$ either. Inversely, we are only interested in strategies which have some intelligence. We are not interested in bounding the regret growth of a policy which simply plays an arm at random each round, forever. Just as the previous class had inapplicable knowledge, this class has inapplicable linear regret.  Surprisingly to the reader, perhaps, these classes aren't so different. Note that any policy that has knowledge of an instance $\nu$ will perform linearly on an instance $\nu'$. Take the policy, $\pi(t)=\argmax_{k\in\mathcal{K}(\nu)}\mu_k(\nu)$ given above, in an instance with another optimal arm it suddenly incurs linear regret. Thus, we would want to simply restrict ourselves to strategies which are sub-linear. For historical reasons, in the literature the condition is made slightly more stringent, requiring sub-polynomial growth. This was the setting chosen by \citet{lai-robbins:1985}, and remains in practice today. Note, however, that this is not a monolithic reverence for the fathers of bandit theory, this requirement is well grounded in fact. Without elaborating on the contents of the next section, there are abstract policies which are sub-polynomial with fractional powers over all instances, and this will be proven. Therefore, we might sensibly limit ourselves to sub-polynomial policies in our analysis, as they are uniformly better than ones which are simply sub-linear.

\begin{definition}[Consistency]
A policy $\pi$ is consistent if for any stochastic bandit in $\mathcal{E}$, and for all $\alpha>0$, we have that:
\[R_n^\nu(\pi) = \mathcal{O}(n^\alpha) \, \text{as}\, n\to\infty\,.\]
\end{definition}

\par This definition, and its justification out of the way, we may proceed to the main result of this section, theorem \ref{thm:LR}. This theorem gives an instance dependent lower bound for the asymptotic regret growth, with remarkable elegance. This result connects the intuitive difficulty of the problem, the number of arms, (the length of the sum), and the gaps between arms with the Kullback-Leibler divergence between the means of each arm. For large gaps, thus extremely different distributions, the resulting regret will be small as the arms are easily distinguished. Even for small gaps, the regret is small if arms are very different, but likely moderate if the arms are quite similar. The divergence really highlights the intrinsic difficulty of the problem.

\begin{theorem}[{Assymptotic Growth Lower-Bound\cite[thm.~1]{lai-robbins:1985}}]\label{thm:LR}
For all consistent policies $\pi$, for all instances $\nu\in\mathcal{E}$, we have:
\[ \liminf_{n\to\infty} \frac{R_n^\nu(\pi)}{\ln(n)} \geq \sum_{i:\Delta_i>0}\frac{\Delta_i}{D_{KL}(P_i\Vert P_{i^*})}\,.\]
\end{theorem}

\begin{proof}
The essence of this proof, as it is in the original document is to focus on bounding the limit inferior of the expected number of samples from sub-optimal arms by consistent policies. We want to show in an instance $\nu\in\mathcal{E}$ that for all $i\in\mathcal{K}$, except for $i^*$ the optimal arm: 
\[\liminf_{n\to\infty} \frac{E[T_i(n)]}{\ln(n)} \geq \frac{1}{D_{KL}(P_i\Vert P_{i^*})}\,.\]
As then the regret decomposition inequality yields our result trivially.

\par In the proof we will consider two instances $\nu$ and $\nu'$ in $\mathcal{E}_K$, and we will prove the result above for the first arm. In the first instance the first arm will have distribution $P_1^{\nu}$ with mean $\mu_1$, and will be suboptimal, with the optimal arm being $P_2^\nu$, with mean $\mu_2$. In the second instance arm one will have $P_1^{\nu'}$ with mean $\lambda>\mu_2$, and all other arms will be identical to $\nu$. We require some technical conditions to move forward with the proof, adapted from \citet{lai-robbins:1985} for greater readability and less dense, updated notation.

\par We will require that the space $\mathscr{S}$ of distributions to which our arms belong is such that for any $\delta>0$, and for any distribution $P_\lambda$ in the space with mean $\mu_\lambda$, there is a distribution $P_\lambda'$ with mean $\mu_\lambda'$ such that $\mu_\lambda<\mu_\lambda'<\mu_\lambda+\delta$. This is the case for any family of distributions where the mean of any member belongs to a real interval, like a gaussian, exponential or poisson distribution. Since it is a technical quirk which does not really influence the theorem, it is given here instead. This condition is very similar to another one, we require, for all $P_\lambda,P_\theta$ with means $\mu_\lambda>\mu_\theta$, and for all $\epsilon>0$, there is $\delta>0$ such that $\vert D_{KL}(P_\theta\Vert P_\lambda)-D_{KL}(P_\theta\Vert P_\lambda')\vert<\epsilon$ when $\mu_\lambda\leq\mu_\lambda'\le\mu_\lambda+\delta$. This is simply continuity of the divergence over $\mathscr{S}$. We will require one final condition, that for arms $i$, and $j$ with $\mu_i > \mu_j$ the Kullback-Leibler divergence from $j$ to $i$ is neither $0$ nor infinite, i.e. that no arms are identical and that they do not violate absolute continuity conditions.

\par Our above requirements, allow us to choose $P_1^{\nu'}$ such that $\lambda>\mu_2$ and such that the following holds:
\begin{equation}
\vert D_{KL}(P_1^\nu\Vert P_1^{\nu'})-D_{KL}(P_1^\nu\Vert P_2^\nu)\vert <\delta D_{KL}(P_1^\nu\Vert P_2^\nu)\,.\label{eq:pfLR-1}
\end{equation}

Since we are restricted to consistent policies, we have, for some $0<\alpha<\delta$:
\[\EE_{\nu'}[n-T_1(n)]=\sum_{i\neq1}\EE_{\nu'}[T_n(i)]=\mathcal{O}(n^\alpha)\,.\]
Now, picking up steam we give successive lower bounds: %RAPID FIRE ROUND!! 
\begin{align*}
\EE_{\nu'}[n-T_1(n)]&\geq \EE_{\nu'}\left[ n - \mathbb{I}\left\{ T_1(n)\ln(n) \leq \frac{(1-\delta)\ln(n)}{D_{KL}(P_1^\nu\Vert P_1^{\nu'})}\right\}\right]\\
\mathcal{O}(n^\alpha)&\geq \EE_{\nu'}\left[ n - \mathcal{O}(\ln(n))\mathbb{I}\left\{ T_1(n) \leq \frac{(1-\delta)\ln(n)}{D_{KL}(P_1^\nu\Vert P_1^{\nu'})}\right\}\right]\\
&\geq  \EE_{\nu'}\left[ (n - \mathcal{O}(\ln(n)))\mathbb{I}\left\{ T_1(n) \leq \frac{(1-\delta)\ln(n)}{D_{KL}(P_1^\nu\Vert P_1^{\nu'})}\right\}\right]\\
&\geq (n - \mathcal{O}(\ln(n))) P_{\nu'}\left( T_1(n) \leq \frac{(1-\delta)\ln(n)}{D_{KL}(P_1^\nu\Vert P_1^{\nu'})}\right)\\
\end{align*}
Now, define $L_m:= \sum_{t=1}^m f(X_t;P_1^\nu)/f(X_t;P_1^{\nu'})$, the empirical divergence estimate where $X_t$ are observed rewards from arm $1$, with $m\leq T_1(n)$, of course. Then, relate both instances $\nu,\nu'$ by defining the event:
\[C_n:=\left\{T_1(n)\leq \frac{(1-\delta)\ln(n)}{D_{KL}(P_1^\nu\Vert P_1^{\nu'})},L_{T_1(n)}\leq (1-\alpha)\ln(n)\right\}\,.\]
One can see that $P_{\nu'}(C_n)=\mathcal{O}(n^{\alpha-1})$. We have shown the first part of $C_n$ to be $\mathcal{O}(n^{\alpha})$, we can see that the second part, while dependent, reduces the probability by at least a factor of $\frac{1}{n}$. Further scrutiny highlights that $C_n$ is a disjoint union of events of the form:
\[� \mathcal{C}:=\bigcap_{i=1}^k(\{T_n(i)=n_i\} )\cap\{L_{n_1}\leq (1-\alpha)\ln(n)\}\,.\]
Under constraints that:
\[\sum_{i=1}^k n_i = n \text{ and } n_1\leq \frac{(1-\delta)\ln(n)}{D_{KL}(P_1^\nu\Vert P_1^{\nu'})}\,.\]
We want to now find a bound relating $P_{\nu'}(C_n)$ and $P_{\nu}(C_n)$. To do so, using the above formulation of $C_n$, notice that:
\begin{align}
P_{\nu'}(C_n)&=\int_{T_1(n)=n_1,\dots,T_n(k)=n_k,L_{n_1}\leq (1-\alpha)\ln(n)}\prod_{t=1}^{n_1} \frac{f(X_t;P_1^\nu)}{f(X_t;P_1^{\nu'})}\de \PP_\nu\notag\\
&=\int_{\mathcal{C}}\exp\left( \ln\left( \prod_{t=1}^{n_1} \frac{f(X_t;P_1^\nu)}{f(X_t;P_1^{\nu'})}\right)\right)\de \PP_\nu\notag\\
&=\int_{\mathcal{C}}\exp\left( -L_{n_1}\right)\de \PP_\nu\notag\\
&\geq \exp(-(1-\alpha)\ln(n))\int_{\mathcal{C}}\de\PP_\nu\label{eq:pfLR-2}\,.
\end{align}
Then, asymptotically, $P_{\nu'}(C_n)\geq n^{\alpha-1}P_\nu(C_n)$, as we desired. Now we present some asymptotical probability results related to $C_n$, which will bring us to the conclusion of our proof up to a small substitution. 
\[\lim_{n\to \infty}P_\nu\left\{\exists i \geq \frac{(1-\delta)\ln(n)}{D_{KL}(P_1^\nu\Vert P_1^{\nu'})} : L_i>(1-\alpha)\ln(n)\right\}=0\,.\]
Then, combining with equation \ref{eq:pfLR-2}, 
\[�\lim_{n\to\infty}P_\nu\left\{�T_1(n) \leq \frac{(1-\delta)\ln(n)}{D_{KL}(P_1^\nu\Vert P_1^{\nu'})} \right\}=0\,.\]
Finally, using our choice of $P_1^{\nu'}$ from equation \ref{eq:pfLR-1}, we have:
\[�\lim_{n\to\infty}P_\nu\left\{�T_1(n) \leq \frac{(1-\delta)\ln(n)}{(1+\delta)D_{KL}(P_1^\nu\Vert P_2^{\nu})} \right\}=0\,.\]
Taking the limiting case for $\delta$, as it can be arbitrarily small, and rearranging terms gives us:
\[\liminf_{n\to\infty} \frac{\EE_\nu[T_1(n)]}{\ln(n)} \geq \frac{1}{D_{KL}(P_1^\nu\Vert P_2^\nu)}\,.\]
This has proven the theorem for an arbitrary sub-optimal arm, specifically $1$, thus it holds for all sub-optimal arms in all environments, and combining with the regret decomposition identity, as stated at the beginning of the proof, gives the regret bound.  
\end{proof}

%
%
%
%
%Weaker version for Gaussian bandits (Csaba)
%
%
%
%



%\par We begin by considering another instance $\nu'\in\mathcal{E}$, where $i^*$ is no longer optimal. We take interest in the sum of the regret over both instances of a consistent policy $\pi$. We will look in particular at the probability of the event that arm $i$ is chosen more or less than half the time in each instance. See that if $T_i(n)$ is greater than $\frac{n}{2}$, then the regret $R_n^\nu(\pi)$ will be reasonably low. Conversely, if $T_i(n)$ is less than $\frac{n}{2}$, then the regret of $R_n^{\nu'}(\pi)$\footnote{Henceforth, we will drop references to the policy $\pi$, as it is unique throughout our analysis, and clutters up already dense formulations of results.} will also be low. This motivates an approach using the Pinsker-type inequality (theorem \ref{thm:PTI}). Let us proceed, with what we have, for all sub-optimal arm $i$:
%\[R_n^\nu\geq \frac{n\Delta_{i}}{2}\PP_\nu\left(T_i(n)\geq \frac{n}{2}\right) \text{, and } R_n^{\nu'}\geq \frac{n(\lambda-\Delta_i)}{2}\PP_{\nu'}\left(T_i(n)<\frac{n}{2}\right)\,.\]
%Thus, letting $\kappa_i:=\min(\Delta_i,\lambda-\Delta_i)$:
%\begin{align*}
%R_n^\nu + R_n^{\nu'} &\geq \frac{n\kappa_i}{2}\left(\PP_\nu\left(T_i(n)\geq \frac{n}{2}\right)+ \PP_{\nu'}\left(\{T_i(n)\geq \frac{n}{2}\}^c\right)\right)\\
%\frac{2(R_n^\nu + R_n^{\nu'} )}{n\kappa_i}&\geq \PP_\nu\left(T_i(n)\geq \frac{n}{2}\right)+ \PP_{\nu'}\left(\{T_i(n)\geq \frac{n}{2}\}^c\right)\,.\\
%\intertext{Applying the Pinsker-type inequality and then reordering the terms to extract the Kullback-Leibler divergence gives us:}
%\frac{2(R_n^\nu + R_n^{\nu'} )}{n\kappa_i}&\geq\frac{1}{2}\exp(-D_{KL}(\PP_\nu\Vert\PP_\nu'))\\
%D_{KL}(\PP_\nu\Vert\PP_\nu')&\geq \ln\left(\frac{n\kappa_i}{4(R_n^\nu + R_n^{\nu'})}\right)\,.
%\intertext{Now applying the divergence decompositions lemma (lemma \ref{lem:DivDecomp}), we have:}
%D_{KL}(P_i^\nu\Vert P_i^{\nu'})\EE_\nu[T_i(n)]&\geq \ln\left(\frac{\kappa_i}{4}\right)+\ln(n)-\ln(R_n^\nu + R_n^{\nu'})\\
%D_{KL}(P_i^\nu\Vert P_i^{\nu'})\frac{\EE_\nu[T_i(n)]}{\ln(n)}&\geq \frac{\ln(\kappa_i)-\ln(4)}{\ln(n)}+1 - \frac{\ln(R_n^\nu + R_n^{\nu'})}{\ln(n)}\,.
%\end{align*}
%The remaining step consists in taking the limit inferior and reordering. First we focus on bounding the term involving the regret, which is where the consistency of $\pi$ is required. As $\pi$ is consistent, there is $p$ arbitrarily small and $C>0$ such that $R_n^\nu + R_n^{\nu'}\leq Cn^p$. This implies $\ln(R_n^\nu + R_n^{\nu'})\leq \ln(C) +p\ln(n)$. We can take the limiting case where $p=0$, to obtain in the limit a vanishing term when divided by $\ln(n)$. Thus:
%\begin{align*}
%\liminf_{n\to\infty} D_{KL}(P_i^\nu\Vert P_i^{\nu'})\frac{\EE_\nu[T_i(n)]}{\ln(n)}&\geq 1 - \liminf_{n\to\infty}  \frac{\ln(R_n^\nu + %R_n^{\nu'})}{\ln(n)}\\
%\liminf_{n\to\infty} D_{KL}(P_i^\nu\Vert P_i^{\nu'})\frac{\EE_\nu[T_i(n)]}{\ln(n)}&\geq 1 - \limsup_{n\to\infty}  \frac{\ln(R_n^\nu + %R_n^{\nu'})}{\ln(n)}\\
%\liminf_{n\to\infty} D_{KL}(P_i^\nu\Vert P_i^{\nu'})\frac{\EE_\nu[T_i(n)]}{\ln(n)}&\geq 1 \\
%\liminf_{n\to\infty}\frac{\EE_\nu[T_i(n)]}{\ln(n)}&\geq \frac{1}{D_{KL}(P_i^\nu\Vert P_i^{\nu'})}\,.
%\end{align*}
%This is as wanted, and as stated at the start of the proof, the regret decomposition identity trivially finishes the proof.
%\end{proof}

%
%
%

\par This theorem only holds for single parameter distributions, but was extended to the multi-parameter case by \citet{burnetas:1996}. To complete this section, we present two simple corollaries applying theorem \ref{thm:LR} to two common cases, first the case of a multi-armed bandit where arms have gaussian reward distribution, and the second to a two-armed bandit, with Bernoulli rewards.

\begin{corollary}[\citethm{banditalgs:7}]\label{cor:gaus}
Given two gaussian distributions with variance $1$ and means $\mu$ and $\mu+\lambda$, their Kullback-Leibler divergence is $\frac{\lambda^2}{2}$. Choosing $\lambda=\Delta_i$ to maximise the bound, we have from theorem \ref{thm:LR}:
\[ \liminf_{n\to\infty} \frac{R_n^\nu(\pi)}{\ln(n)} \geq \sum_{i:\Delta_i>0}\frac{2}{\Delta_i}\, .\]
\end{corollary}

\begin{corollary}[{\citethm[thm.~2.2]{bubeck:2012}}]\label{cor:bern}
In a stochastic bandit, with rewards from Bernoulli distributions, the bound from theorem \ref{thm:LR} becomes by Pinsker's inequality:
\[\liminf_{n\to\infty} \frac{R_n^\nu(\pi)}{\ln(n)} \geq \sum_{i:\Delta_i>}\frac{1}{2\Delta_i}\,.\]
\end{corollary}

\par As our first important result in the analysis of the stochastic bandit problem, we derived the optimal asymptotic growth rate for any policy. This will allow us to compare limiting performance upper bounds for our algorithms to a benchmark. The proof was also our first intense foray into the details for which we developed tools in chapter \ref{chap:FP}. It is somewhat different from the rest of the proofs in this thesis, as it maintains the older methods of \citet{lai-robbins:1985}, while updating the contents and expanding on them to facilitate understanding by the reader. Several other proofs exist, such as the one by \citet{bubeck:2012}, or the one by \citet{banditalgs:7}, but both restrict themselves to a sub-class of bandit problems, and prove respectively corollaries \ref{cor:bern} and \ref{cor:gaus} only, instead of the general case. The general case proof as given is more complex, and most likely the most complex proof in this thesis, but it is rewarding in the elegance of its result. The next section will give a finite time quantity to use as a regret benchmark.

%
%
\subsection{Minimax Regret bound}
%
%

%sub opt gaps must be bounded above, o.W we can have one arm pull fuck up the whole regret by being larger than the rest 

%allows us to compare algorithms across all problems

%Canonical bandits etc. 

\par The {\em minimax} regret is a fundamental quantity of the difficulty of the stochastic bandit problem. A term borrowed from game theory, it represents the smallest regret achievable in the worst case problem by any policy or algorithm. If the particular instance in $\mathcal{E}_K$ we are working with is reasonably well behaved, we can achieve lower regret, but if we are in the worst possible instance, the minimax is a meaningful lower bound for the best possible performance of any algorithm. To prove this bound we must first introduce a new framework for describing the stochastic bandit setting, and prove a lemma on the KL-divergence. 

\par This new canonical bandit model is a new angle from which to view the theory of multi-armed bandits. It is not the most obvious, but it can be very useful, and will be used in our proof of the minimax bound, thus we introduce it here. First we must define a new measurable space, in terms of a notion of {\em history}. It is natural, therefore, to begin with the following definition.

\begin{definition}\label{def:hist}[History]
The collection of all past actions and rewards is called the history of the problem. There are several ways to consider the history, we give here three notations to clearly differentiate them. We denote the random variable $H_n:= (A_1,X_1, \dots, A_n, X_n)$, and a realisation $h_n:=(a_1,x_1,\dots, a_n,x_n)$. We are also interested in the sample space to which all $H_n$ belong, the {\em history space}, which we denote $\mathcal{H}_n$. See that this sample space is clearly composed of $n$ pairs each consisting of an integer less than $K$, the arm corresponding to the action, and a real valued reward. Thus, formally $H_n:=(\llbracket K\rrbracket \times \RR)^n$. Do note, that this definition implies we henceforth hold $K$ fixed.
\end{definition}

\par To work in the history space $\mathcal{H}_n$, we would need to extend it to a measurable space, so that we can define measures rigorously in $\mathcal{H}_n$. To do so, we will use the Lebesgue $\sigma$-algebra, based on the Lebesgue measure mentioned in section \ref{subsec:Measures}. The Lebesgue $\sigma$-algebra on $S$, $\mathcal{L}(S)$ is the $\sigma$-algebra of all sets in $S$ measurable with respect to the Lebesgue measure. In our case, we take $\mathcal{L}_n:=\mathcal{L}(\mathcal{H}_n)$, which is equivalent to restricting the Lebesgue sigma-algebra of $R^{2n}$ to $\mathcal{H}_n$.

\par See that now, a single realisation from $\mathcal{H}_n$ gives us the entirety of the contents of an $n$ round run against a $K$-armed bandit. It is easily seen in fact that $\mathcal{H}_n$ consists precisely of any possible course of proverbial events in an $n$-round, $K$-armed, bandit. From this realisation we can see that the interaction of any instance $\nu\in\mathcal{E}_K$ and any policy $\pi$ is in $\mathcal{H}_n$. Since we now have a valid measurable space, the next logical step is to look at measures on this space. Consider the random variables $A_t$, $X_t$ for $t\leq n$, on the canonical bandit $(\mathcal{H}_n,\mathcal{L}_n)$, see that for all $h_n\in\mathcal{H}_n $, trivially $A_t(h_n)=a_t$ and $X_t(h_n)=x_t$. This means that in the canonical bandit model, these are fixed and no longer random. More interestingly still, they do not depend on an instance or policy. 

\par In this new canonical model, we now set out to prove two lemmas, \ref{lem:dPnu} and then \ref{lem:DivDecomp}, which will be used in the proof of our main result on the minimax regret. 

\begin{lemma}[\citethm{banditalgs:6}]\label{lem:dPnu}
In a $K$-armed bandit instance $\nu$, with arms with distribution $P_i$ for $i\in\llbracket K\rrbracket$, we have:
\[d\mathbb{P}_\nu (h_n) = \prod_{t=1}^n P(a_t\vert a_s,x_s : s\leq t,\pi) \de P_{a_t}(x_t)\de\rho(a_t)\, .\]
Where $P_\pi(a_t\vert a_s,x_s : s\leq t)$ denotes the probability that under policy $\pi$ the next action taken given the history up to $t-1$ is $a_t$, and $\rho(a_t)$ is the counting measure. 
\end{lemma}

\begin{proof}
Expanding, by conditioning, we have:
\begin{align*}
\de\PP_\nu(h_n)&= \de\PP_\nu(h_n\vert h_{n-1})\de\PP_\nu(h_{n-1})\\
&=\prod_{t=1}^n\de\PP_\nu(h_t\vert h_s :s<t)\\
&=\prod_{t=1}^n\de A_t(h_t\vert h_s :s<t)\de X_t(h_t\vert h_s :s<t)\\
&=\prod_{t=1}^n \de P(a_t\vert a_s,x_s : s<t,\pi)\de P_{a_t}(x_t)\de\rho(a_t)\,.
\end{align*}
The last step follows by defining the $\de P(\cdot)$ as the Radon-Nikodym derivatives with respect to an arbitrary measure, in this case, $\rho(a_t)$, the counting measure of $\llbracket K \rrbracket$. 
The proof is complete but we can go one step further and expand the Radon-Nikodym derivative $p_{a_t}$of the $\de P_{a_t}$ with respect to an arbitrary measure $\lambda$ which dominates all of them, to obtain:
\[\de\PP_\nu(h_n)=\prod_{t=1}^n \de P(a_t\vert a_s,x_s : s<t,\pi)\de\rho(a_t)\de\lambda(x_t)p_{a_t}(x_t)\,.�\]
\end{proof}

\begin{lemma}[\citethm{banditalgs:6}]\label{lem:DivDecomp}%[Divergence decomposition]
For two instances $\nu,\nu'\in\mathcal{E}_K$, with arms $P_i$ and $P_i'$ respectively, under the same policy $\pi$, we have:
\[D_{KL}(\mathbb{P}_\nu \Vert \mathbb{P}_\nu') = \sum_{i=1}^K \mathbb{E}_\nu[T_i(n)]D_{KL}(P_i\Vert P_i')�\,.\]

\end{lemma}

\begin{proof}
We consider two instances, $\nu,\nu'\mathcal{E}_K$, with arm measures $P_i$ and $P_i'$, and assume $P_i\ll P_i'$. We devise the measure $\lambda:= \sum_{i\in\mathcal{K}}P_i + P_i'$. This means we can define Radon-Nikodym derivatives of all arms with respect to $\lambda$, denoted $p_i$ and $p_i'$ for arms $P_i$ and $P_i'$ respectively. Recall, that by definition:
\[ D_{KL}(\PP_\nu \Vert \PP_{\nu'}):= \int \ln\left( \frac{\PP_\nu}{\PP_{\nu'}}\right)\de\PP_\nu = \EE_\nu\left[ \ln\left( \frac{\PP_\nu}{\PP_{\nu'}}\right)\right]\,.\]
Using lemma \ref{lem:dPnu}, followed by the chain rule, noting that all policy, $\lambda$, and $\rho$ terms cancel:
\[�\ln\left( \frac{\PP_\nu}{\PP_{\nu'}}\right) = \sum_{t=1}^n \ln\left( \frac{p_{A_t}(X_t)}{p_{A_t}'(X_t)}\right)\,.\]
Now we apply the tower property and the fact that $p_{a_t}\de\lambda=\de P_{a_t}$ to rewrite the divergence as:
\[�\EE_\nu\left[ \ln\left( \frac{p_{A_t}(X_t)}{p_{A_t}'(X_t)}\right)\right] =\EE_\nu\left[ D_{KL}(P_{a_t}\Vert P_{a_t})\right] \,.\]
Combining both, we have:
\begin{align*}
D_{KL}(\PP_\nu \Vert \PP_{\nu'})&=\sum_{t=1}^n \EE_\nu\left[ D_{KL}(P_{A_t}\Vert P_{A_t})\right]\\
&=\sum_{i\in\mathcal{K}} \EE_\nu\left[ \sum_{t=1}^n D_{KL}(P_{A_t}\Vert P_{A_t}-)\mathbb{I}\{�A_t=i\}\right]\\
&=\sum_{i\in\mathcal{K}} \EE_\nu[T_i(n)]D_{KL}(P_i\Vert P_i')\,.
\end{align*}
\end{proof}


\par Lemma \ref{lem:dPnu} is not used directly in theorem \ref{thm:minimax}, but rather in the proof of lemma \ref{lem:DivDecomp}, which in turn is used in the proof of the theorem. This semantic aside is an artefact of the organisation of the proof and of the lemmas, but both will be required for our proof of the next theorem, which gives us a bound for the instance-independent minimax-regret.

%%%%%%%%%%% Thm and Proof of minimax

\begin{theorem}[Minimax Regret Bound\cite{banditalgs:6}]\label{thm:minimax}
The minimax pseudo-regret of the class $\mathcal{E}_K$, given $n>K-1$ is:
\[\bar{R}^*_n(\mathcal{E}_K) \geq \frac{1}{27}\sqrt{n(K-1)}\,.\]
\end{theorem}

\begin{proof}
Let $\mathcal{G}_K$ denote the class of bandits within $\mathcal{E}_K$ where specifically the distributions of the arms are exactly gaussian with unit variance. We will prove this result on $\mathcal{G}_K$, rather than $\mathcal{E}_K$. This slight lack of rigour is justified by the assumed sub-gaussianity of rewards in $\mathcal{E}_K$. Intuitively, if the arms have lighter tails than a gaussian distribution, then distinguishing arms should be easier, meaning that the minimax regret is presumably higher over the class $\mathcal{G}_K$ than of bandits with stricktly lighter reward tails.
\par  Throughout, let $\pi$ be a given policy. The idea behind the proof is to make $\pi$, which performs well on one bandit fail on another by designing the second environment entirely to trap the agent performing $\pi$. We begin with the first environment. Let $0\leq r\leq \frac{1}{2}$, the first environment in $\mathcal{G}_K$ has means $r$ for arm $1$ and $0$ otherwise. In this environment we denote $\mathbb{P}_1$ the distribution on the canonical bandit $(\mathcal{H}_n,\mathcal{L}_n)$.
\par For the second environment we take the optimal arm $i$ to be arm least taken by $\pi$ in $1$: $i=\argmin_{j\neq1}\mathbb{E}_1[T_j(n)]$. We can now engineer a loss on this instance by taking the mean rewards from the first case and changing the reward for arm $i$ to $2r$. It remains now to combine these to show a bound. First we will rewrite both regrets $\bar{R}_n^1(\pi)$ and $\bar{R}_n^2(\pi)$, then apply the Pinsker-type inequality (theorem \ref{thm:PTI}) and finally lemma \ref{lem:DivDecomp}.

\begin{align}
\bar{R}_n^1+ \bar{R}_n^2 &> \frac{nr}{2}\left( \mathbb{P}_1(T_1(n)\leq \frac{n}{2}) + \mathbb{P}_2(T_1(n)> \frac{n}{2})\right)\notag\\
&>\frac{nr}{4}\exp\left(-D_{KL}(\mathbb{P}_1\Vert \mathbb{P}_2)\right)\,.\label{eq:DKL}
\end{align}
Now, we use lemma \ref{lem:DivDecomp} to bound $D_{KL}(\mathbb{P}_1\Vert \mathbb{P}_2)$.
\[ D_{KL}(\mathbb{P}_1\Vert \mathbb{P}_2)= 2r^2\mathbb{E}_1[T_i(n)]\, . \]
Here we use the fact that $\sum_{j>1}\mathbb{E}_1[T_j(n)]\leq n \Rightarrow \mathbb{E}_1[T_i(n)]\leq \frac{n}{K-1}$ to further complete our bound, which upon replacement in \ref{eq:DKL} yields:
\[ \bar{R}_n^1+ \bar{R}_n^2 \geq \frac{nr}{4}\exp\left(-\frac{2nr^2}{K-1}\right)\,.\]
Finally, to obtain the result we take $r=\sqrt{\frac{K-1}{4n}}$, which means $n>K-1$ as $r<\frac{1}{2}$.
\begin{align*}
 \bar{R}_n^1+ \bar{R}_n^2 &\geq \frac{1}{8}\sqrt{n(K-1)}\exp\left(-\frac{1}{2}\right)\\
 &\geq \sqrt{n(K-1)}\times \frac{1}{8} \times \frac{29}{48}\\
 &\geq \frac{29}{384}\sqrt{n(K-1)}\,. 
\intertext{The second line follows from the power series expansion of $\exp(-\frac{1}{2})$ up to the fourth term. Finally, from $\bar{R}_n^*=\max\{\bar{R}_n^1,\bar{R}_n^2\}\geq \frac{1}{2}(\bar{R}_n^1+\bar{R}_n^2)$, we recover:}
\bar{R}_n^*&\geq \frac{29}{768}\sqrt{n(K-1)}\, . 
\end{align*}
 The largest simple fraction less than this is $\frac{1}{27}$. This change only tidies up the formula. 
\end{proof}


\par This theorem gives us guarantees on the worst-case regret of any algorithm over any possible stochastic bandit environment. In many instances the regret will be less than this bound, but unavoidably, some instances will suffer regret of $\Omega(\sqrt{nK})$. This does not mean however that we should aim for an algorithm which achieves $\mathcal{O}(\sqrt{nK})$ everywhere. Indeed this algorithm might perform terribly on nicer instances, where logarithmic asymptotic growth may be reached well before $n$. There is a design trade-off between approaching the optimal minimax regret in the worst-case and maintaining optimal regret in easier instances.  We will discuss this in more detail when we review algorithms in the next chapter and we will then nuance this apparent trade-off. 
%MOSS bubeck
